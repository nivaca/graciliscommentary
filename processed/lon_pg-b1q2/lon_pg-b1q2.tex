
        %this tex file was auto produced from TEI by lombardpress-print on 2015-12-24T14:30:43.727-05:00 using the  file:/Users/JCWitt/Desktop/lombardpress-print/lbp-latex-critical.xsl 
        \documentclass[twoside, openright]{article}
        
        % etex package is added to fix bug with eledmac package and mac-tex 2015
        % See http://tex.stackexchange.com/questions/250615/error-when-compiling-with-tex-live-2015-eledmac-package
        \usepackage{etex}
        
        %imakeidx must be loaded beore eledmac
        \usepackage{imakeidx}
        
        \usepackage{eledmac}
        \usepackage{titlesec}
        \usepackage [english]{babel}
        \usepackage [autostyle, english = american]{csquotes}
        \usepackage{geometry}
        \usepackage{fancyhdr}
        \usepackage[letter, center, cam]{crop}
        
        
        \geometry{paperheight=10in, paperwidth=7in, hmarginratio=3:2, inner=1.7in, outer=1.13in, bmargin=1in} 
        
        %fancyheading settings
        \pagestyle{fancy}
        
        %section headings
        \titleformat{\chapter}[display]
        {\normalfont\huge}{}{20pt}{\Huge}
      
        %quotes settings
        \MakeOuterQuote{"}
        
        %title settings
        \titleformat{\section} {\normalfont\scshape}{\thesection}{1em}{}
        \titlespacing\section{0pt}{12pt plus 4pt minus 2pt}{12pt plus 2pt minus 2pt}
        \titleformat{\chapter} {\normalfont\Large\uppercase}{\thechapter}{50pt}{}
        
        
        %eledmac settings
        \foottwocol{B}
        \linenummargin{outer}
        \sidenotemargin{inner}
        
        %other settings
        \linespread{1.1}
        
        %custom macros
        \newcommand{\name}[1]{\textsc{#1}}
        \newcommand{\worktitle}[1]{\textit{#1}}
        
        
        \begin{document}
        \fancyhead[RO]{Book I, Question 2 [London Transcription]}
        \fancyhead[LO]{0.1.0}
        \fancyhead[LE]{Peter Gracilis}
        \chapter*{Book I, Question 2 [London Transcription]}
        
         
        \beginnumbering
         \section*{Quaestio 2 [London Transcription]} 
        \pstart
        \ledsidenote{\textbf{1}}
        Veteris ac novae legis etc praemissio prohemioprooemio sequitur tractatus in quo \name{magister}\index[persons]{Peter Lombard} duo facit nam primo praemittit quaedam praeambula 2o incipit tractanda 2a incipit in principio 2ae distinctionis prima dividitur in 3 partes principales nam in prima materias librorum per quasdam distinctiones abinvicem distinguit et separat in 2a movet quasdam quaestiones et eas per tractat in 3a ponit quasdam resumptiones et circa dicta epilogat  2a ibi cum autem homines 3a ibi ergo ominum qua dicta sunt prima pars adhuc dividitur in duas quia primo \name{magister}\index[persons]{Peter Lombard} inquiritur materias librorum divisive 2o diffinitive 2a ibi fruit etc 2a pars principalis dividitur in tres secundum quod \name{magister}\index[persons]{Peter Lombard} tres movet quaestiones prima utrum homo possit licite frui se  2a utrum deus fruatur homine 3a utrum homo licite fruatur virtute 2a ibi ibi sed cum deus 3a ibi hic considerandum est  quaelibet harum partium dividitur in duas quia ponit movet quaestionem 2o solvit seu ponit quaestionis solutionem partes patebunt  3a pars principalis epilogia dividitur in tresduas quia primo epilogat  2o modum dicendorum insipiat et ad dicenda se continuat 2a ibi de quibus omnibus etc et haec sit divisio in generali 
        \pend
     
        \pstart
        \ledsidenote{\textbf{2}}
        Utrum aliquod suppositum procreatae entitatis sit fruibile obiectum ordinatae voluntatis  vel sic utrum sicut voluntas creata est relatae dilectionis subiectum sic sola trinitas beata sit ordinatae fruitionis obiectum 
        \pend
     
        \pstart
        \ledsidenote{\textbf{3}}
        quod non quia non solum voluntas est dilectionis subiectum ergo antecedens probatur quia dilectio est cognitio sed cuiuslibet cognitionis creatae intellectus est subiectum ergo  maior patet quia non est maius inconveniens ponere dilectionem esse cognitionem quam dicere assensum esse apprehensionem  tum quia sequitur quod aliquis esset beatus et non videret deum patet quia si sint duo actus deus posset unum sine alio conservare
        \pend
     
        \pstart
        \ledsidenote{\textbf{4}}
        2o non sola trinitas est ordinatae fruitionis obiectum ergo probatur antecedens tum quia quaelibet creatura est infinite diligibilis ergo et ordinatae fruibilis patet consequentia quia si non est fruibilis ergo est tam diligibilis quod non plus diligibilis primum antecedens probatur per \name{hugolinum}\index[persons]{Hugolino of Orvieto} libro primo conclusione 3a articulo primo  Item ipsum probat \name{fastinus}\index[persons]{Fascinus of Asti} in suo primo quia deus quaelibet creaturam diligit infinite
        \pend
     
        \pstart
        \ledsidenote{\textbf{5}}
        3o si angelus sathanaesatanae transformaret se in speciem lucis creatura posset eo licite frui ergo antecedens patet  \name{magistrum}\index[persons]{Peter Lombard} libro primo distinctione 30 et in candem 2a quaestione prima ubi dicitur quod in tali casu credens dyabolum esse christum licite posset adorare eum nec talis error esset periculosus
        \pend
     
        \pstart
        \ledsidenote{\textbf{6}}
        In oppositum est \name{magister}\index[persons]{Peter Lombard} in distinctione praesenti  allegans \name{augustinum}\index[persons]{Augustinus} primo \worktitle{de doctrina christiana}\index[works]{De doctrina christiana} capitulo 20 res autem quae nos beatos faciunt et quibus fruendum est sunt pater et filius et spiritus sanctus
        \pend
     
        \pstart
        \ledsidenote{\textbf{7}}
        prima conclusio de supposito licet omnis dilectio deppendeatdependeat causaliter a cognitione tamen quaelibet obiecti apprehensio vel cognitio cum voluntatis libertate sufficit dilectionem causare  prima probatur quia si non sequitur quod dilectio posset poni seu elici naturaliter a voluntate seclusa omni cognitione consequens est falsum \ledsidenote{L15v} quia tunc voluntas posset diligere in infinitum contra \name{augustinum}\index[persons]{Augustinus} in libro 8 2 10 \worktitle{de trinitate}\index[works]{De Trinitate} patet consequentia quia positis omnibus causis ad productionem ad productionem alicuius effectus requisitis omni alio secluso talis effectus posset naturaliter poni in esse 2a pars probatur quia quia si sola obiecti cognitio etc sequitur quod stante iudicio vel apprehensione alicuius obiecti sub ratione bonimali seclusa omnia existentia vel apparentia bonitatis voluntas posset tale obiectum velle vel diligere consequentia nota sed consequens est contra \name{philosophum}\index[persons]{Aristotle} et \name{commentatorem}\index[persons]{Averroes} primo \name{ethicorum}\index[persons]{} quia omnia bonum appetunt
        \pend
     
        \pstart
        \ledsidenote{\textbf{8}}
        primum corollarium quod nulla dilectio vel volitio est essentialiter cognitio vel formaliter apprehensio patet quia causa et effectus distinguuntur essentialiter et antecedens patet per conclusionem quia dilectio causaliter deppendetdependet a cognitione et non in genere causae formalis materialis nec finalis ergo efficientis
        \pend
     
        \pstart
        \ledsidenote{\textbf{9}}
        2m corollarium si dilectio respectu alicuius obiecti sit in voluntate neccessenecesse est respectu eiusdem cognitionem in intellectu esse vel praefuisse loquendo naturaliter et de communi lege patet   conclusionem quia effectus praesupponit suam causam
        \pend
     
        \pstart
        \ledsidenote{\textbf{10}}
        3m corollarium si deus potest vicem cuiuslibet causae 2ae supplere ipse potest in voluntate dilectionem causare
        \pend
     
        \pstart
        \ledsidenote{\textbf{11}}
        2a conclusio quam impossibile est entitatem creatam esse intrinsece bonitatis infinitae et illimitatae tam est impossibile alicamaliquam talem habere rationem intrinsecam diligibilitatis infinitae patet prima quia sola natura increata est et esse potest bonitatis illimitate intrinsece ergo patet antecedens per \name{augustinum}\index[persons]{Augustinus} \worktitle{super genesepsalmos}\index[works]{} 134 dicens quod solus deus est summe bonus cui nihil boni simpliciter de esse potest  confirmatur quia aliter creatura posset esse indeppendensindependens 2a pars probatur quia quodlibet creatum bonum est insufficiens sibi et indigens alio bono ergo nullum est infinite diligibile ex ratione sua intrinseca patet consequentia quia non est intrinsece maioris diligibilitatis quam eius bonitas intrinseca
        \pend
     
        \pstart
        \ledsidenote{\textbf{12}}
        primum corollarium nullum diligens rem incomplexe ordinate debet latitudinem suae dilectionis extendere ultra valorem intrinsecum rei amatae patet  \name{augustinum}\index[persons]{Augustinus} 9 \worktitle{de trinitate}\index[works]{De Trinitate} capitulo 9 ubi ostendit quod quanto res est minus bona quam deus tanto minus est diligi debet et concludit in fine quod quaelibet res quae non est deus debet infinite minus diligi quam deus
        \pend
     
        \pstart
        \ledsidenote{\textbf{13}}
        2m corollarium nullus diligens deum minus quam creaturam entitate fruitur ipso seu diligit ordinante patet quia aequalis perversitas est frui utendis et uti fruendis
        \pend
     
        \pstart
        \ledsidenote{\textbf{14}}
        3m corollarium quam repugnat aliquem deo ordinate uti tam repugnat creaturam iuste creaturae frui patet quia null creaturae correspondet tanta bonitas quanta bonitas est dei ergo nec tanta diligibilitas  nota quod ista est probatio praecedentis corollarii sequitur corollarie omnis ponenes deum esse posse demonstrari habet ponere ipsum esse quod super omnia debet amari patet ex virtute de vocabuli nam deus est id quod melius cogitari non potest secundum regulam \name{anselmi}\index[persons]{Anselm} sed quanta est intrinseca bonitas rei tanta est eius diligibilitas per dicta igitur
        \pend
     
        \pstart
        \ledsidenote{\textbf{15}}
        3a conclusio sicut sola trinitas increata est ordinatae fruitionis ratio obiectiva sic eadem sub unica ratione est creatae ymaginisimaginis ad se fruendum motiva prima patet  \name{augustinum}\index[persons]{Augustinus} primo \worktitle{de doctrina christiana}\index[works]{De doctrina christiana} ubi prius allegatum Item primo \worktitle{Confessionum}\index[works]{Confessions} ad te domine nos fecisti et inquietum est cor nostrum donec requiescat in te confirmatur quia nulla creatura sibi sufficit ad esse nec ad bene esse ergo nec sufficit alteri et sic in creatura non potest esse quietatio ultimata et finalis  2a pars probatur quia si non hoc esse quia \ledsidenote{L16r}moveret intellectum sub ratione vis et voluntatem sub ratione boni sed hoc non impedit quia omnino est eadem ratio bonitatis et veritatis immense sed trinitas immensa non movet partes ymaginisimaginis ad se fruendum nisi sub ratione veritatis et bonitatis immense igitur unica ratione
        \pend
     
        \pstart
        \ledsidenote{\textbf{16}}
        primum corollarium sicut trinitas increata sub unica ratione movet et ostendit se beatifice creaturae ymaginiimagini sic non stat creatam ymaginemimaginem memoriam sine intellectiva beatifice elevari prima patet conclusionem 2a probatur tum quia est unica motio  tum quia motum seu receptivum est unum et idem scilicet memorativa intellectiva potentiae ut videtur sentire \name{augustinus}\index[persons]{Augustinus} 9 \worktitle{de trinitate}\index[works]{De Trinitate} capitulo 3o  per hoc tamen nolo asserere quin homo posset deum intelligere clare et non diligere quia forte stat quod videat clare et non diligat sed non beatifice
        \pend
     
        \pstart
        \ledsidenote{\textbf{17}}
        2m corollarium quam repugnat creatam ymaginemimaginem beatifice frui persona divina sine essentia tam repugnat ei frui essentia et non qualibet perfectione sua intrinseca probatur quia omnis perfectio divina est formaliter et realiter persona et essentia et econverso
        \pend
     
        \pstart
        \ledsidenote{\textbf{18}}
        3m corollarium sicut beata trinitas intensius et remissius ad sui fruitivam perceptionem immutat vitaliter sic quodlibet obiectum vitaliter perceptivum ad sui perceptionem concurrit causaliter prima pars probatur quia libere movet potentiam creatam et totam sui perceptionem libere efficit cum sit agere ad extra  Ex quo posset inferri quod licet causa totalis et univoca genus speciei non possit excedere tamen partialis et aequivoca ultra perfectionem naturae proprie potest efficere prima patet in multis locis philosophiae 2a per dicta quia res materialis seu obiectum causaliter concurrit ad cognitionem quae est immaterialis ex dictis sequitur pars affirmativa quaestionis
        \pend
     
        \pstart
        \ledsidenote{\textbf{19}}
        Contra primam conclusionem et eius corollaria arguitur quia alicaaliqua volitio et dilectio formaliter est cognitio ergo antecedens probatur quia si destrueretur in mente cognitio remanente dilectione respectu obiecti dilecti quaeritur autem per talem dilectionem obiectum cognosceretur aut non si sic habetur propositum si non sequitur quod in cognitum diligit contra \name{augustinum}\index[persons]{Augustinus} Item ad hoc est prima ratio principalis facta ad oppositum quaestionis 
        \pend
     
        \pstart
        \ledsidenote{\textbf{20}}
        2o ubicumque est unitas obiecti moventis sub unica ratione ibi est unitas actuum sed in operatione intellectus et voluntatis saltem in fruitione patriae est unitas obiecti moventis sub unica ratione ergo ibi est veritas actuum et per consequens cognitio et dilectio sive diligere et cognoscere erit idem actus non plurificatus minor patet  2am partem 3ae conclusionis maior patetpatet  \name{philosophum}\index[persons]{Aristotle} 2o \worktitle{de anima}\index[works]{de Anima}
        \pend
     
        \pstart
        \ledsidenote{\textbf{21}}
        Contra 2am cognitionem et eius corollaria arguitur primo quaelibet creatura licite potest infinite plus diligi quam nunc diligitur igitur
        \pend
     
        \pstart
        \ledsidenote{\textbf{22}}
        2o quia deus et omnis multitudo creata sunt bonitas infinite diligibilis et diligibilis infinite et non sunt deus sed aliud a deo ergo aliud a deo est fruendum ordinate
        \pend
     
        \pstart
        \ledsidenote{\textbf{23}}
        3o deus diligit qualibet rem infinite et immense ergo quaelibet res est diligibilis infinite antecedens patet quia deus diligit tantum et taliter qualiter ipse est cum suum diligere et sua dilectio et suus modus diligendi sicut idem cum ipso realiter qui est infinitus et infinite
        \pend
     
        \pstart
        \ledsidenote{\textbf{24}}
        4o contra 2m corollarium homo fruentius licite \ledsidenote{L16v}potest frui in proximum quam in deum ergo licite plus diligere consequentia patet cum sit in aliud frui sit diligere antecedens patet quia quis licite potest plus et fruentius plorare mortem amici quam mortem christi et plus temporalia dampnata quam peccata commissa 
        \pend
     
        \pstart
        \ledsidenote{\textbf{25}}
        Contra 3am conclusionem arguitur primo sic cuiuslibet creaturae rationalis potentiae receptiva est finita et limitata ergo alicaaliqua limitata bonitas potest eam beatificare et finaliter quietare antecedens probatur quia nulla creatura est maior perceptibilitas seu capacitas quam entitas
        \pend
     
        \pstart
        \ledsidenote{\textbf{26}}
        contra idem 2o deus solum beatificat finite quamlibet creaturam beatam ergo bonitas finita potest influxum eius obiectivum in perceptione creaturae suplere consequentia patet quia omni finito data potest dari aliud finitum aequale vel melius in perfectione
        \pend
     
        \pstart
        \ledsidenote{\textbf{27}}
        contra 2am partem eiusdem conclusionis obiectum intellectus et voluntatis non possunt  movere sub eadem ratione ergo antecedens probatur quia obiectum intellectus est intelligible ut intelligibile obiectum vero voluntatis volibile ut sic sed illa differunt ratione quod probatur primo quia aliquis potest hodie esse volens et intelligens unum et cras intelligere idem sed non velle
        \pend
     
        \pstart
        \ledsidenote{\textbf{28}}
        2o deus hodie intelligit hominem quem non vult diligere et heri ipsum intellexit et dilexit sed illa diversitas non est in deo ergo in rationibus obiectalibus
        \pend
     
        \pstart
        \ledsidenote{\textbf{29}}
        3o deus malum culpae intelligit et tamen non vult ipsum ergo
        \pend
     
        \pstart
        \ledsidenote{\textbf{30}}
        contra eadem partem 2o arguitur principaliter sic sequitur quod quemlibet pars ymaginisimaginis sit elicitiva fruitionis consequens est falsum quia fruitio est actus solius voluntatis cum sit dilectio seu diligere aliquam rem propter se ipsam consequentia patet quia si una non elicit ergo una movetur ad eliciendum et alia non ergo non unica ratione movet quamlibet illarum quod est oppositum illius partis
        \pend
     
        \pstart
        \ledsidenote{\textbf{31}}
        Ad primum potest negari primo antecedens et ad probationem dicitur quod si remaneret dilectio sine notitia quod tunc vel deus suplet vicem notitiae ut dicebatur in uno corporo  vel illud diligeret mediante cognitione ostensiva praecedente
        \pend
     
        \pstart
        \ledsidenote{\textbf{32}}
        2o potest dici quod licet dilectio possit esse quemdam experimentalis notitia seu perceptio non tamen potest esse nostra cognitio intellectiva seu iudicativa vel indicativa et per hoc potest solvi prima ratio in oppositum quaestionis negando antecedens et ad eius primam probationem dicitur quod non est inconveniens ponere assensum esse apprehensionem quia ad intellectum pertinet assentire et apprehendere sed dicere dilectionem esse cognitionem saltim indicativam est inconveniens
        \pend
     
        \pstart
        \ledsidenote{\textbf{33}}
        Ad 2am probationem dicitur quod nullus potest esse perfecte beatus nisi deum videat ideo negatur consequentia sicut dicit \name{magister honofrius}\index[persons]{} de hoc vide \name{gregorium}\index[persons]{Gregory of Rimini} libro primo distinctione prima articulo 2o quaestione 2a
        \pend
     
        \pstart
        \ledsidenote{\textbf{34}}
        Ad 2m diceret \name{alphonsus}\index[persons]{Alonso Vargas of Toledo} libro primo distinctio prima negando minorem quia tenet quod in fruitione patriae anima non se habet active sed pure passive
        \pend
     
        \pstart
        \ledsidenote{\textbf{35}}
        2o posset dici quod licet proprie loquendo una potentia moveatur causaliter effective ad eliciendum actum tamen hoc motio non est formalis et obiectiva sed solum est ad percipiendum modo hic loquor de motione formali perceptiva et obiectali et non de alia 
        \pend
     
        \pstart
        \ledsidenote{\textbf{36}}
        3o posset dici quod quaelibet potentia beatificalis in illa unica ratione habet seu potest sumi propria obiectum reperire quia eius motio obiectalis est libere praesentativam quod non est in aliis obiectis quae naturaliter se obiciunt et movent
        \pend
     
        \pstart
        \ledsidenote{\textbf{37}}
        Ad 3m primo diceret \name{hugonus}\index[persons]{Hugolino of Orvieto} distinctione prima primi quaestione 3a concedo antecedens sic quod ab infinitis voluntatibus plus potest in infinitum diligi sed non ab una voluntate vide ibi Aliter dicitur negando \ledsidenote{L17r}hanc consequentiam haec creatura est infinite diligibilis ergo est fruibilis nisi addatur ex valore intrinseco rei
        \pend
     
        \pstart
        \ledsidenote{\textbf{38}}
        Ad 4m conceditur antecedens sic intelligendo quod nihil aliud a deo deum non includens vel cuius deus non est pars est diligibile infinite modo illa multitudo includit deum per positum 2o dici potest quod illa multitudo est diligibilis pluribus dilectionibus quam deus sed non plus intensive seu plura bona debet homo velle tali multitudini quam dei non tamen maius bonum sic posset dici quod christus est plus diligendus quam pater id est pluribus modis quia passus pro nobis
        \pend
     
        \pstart
        \ledsidenote{\textbf{39}}
        Ad 5m posset negari antecedens 2o posset negari haec consequentia quaelibet res est diligibilis infinite ex idem intrinseca ergo est fruibilis quia requireretur quod ratione suae intrinsece bonitatis esset intrinsece infinite diligibilis Item posset dici quod licet sit diligibilis infinite non tamen est diligibilis infinite
        \pend
     
        \pstart
        \ledsidenote{\textbf{40}}
        Ad aliud dicitur quod aliquid frui dicitur dupliciter primo modo cordiali affectu 2o corporali effectu 3o sensuali conatu sed 2o et 3o modis sumendo frui conceditur antecedens et negatur consequentia
        \pend
     
        \pstart
        \ledsidenote{\textbf{41}}
        Ad aliud contra 3am conclusionem diceret \name{thomas argentis}\index[persons]{Thomas of Strasbourg} quod anima est infinitae capacitatis potentia obiectali non naturali vide quaestione 3a prologi articulo primo  Aliter posset dici quod anima est infinitae capacitatis et infinitae perceptitatis licet non sit potentia perceptiva infinita nec infinite et posset negari consequentia \name{clienton}\index[persons]{Richard of Kilvington}  quaestione 3 dicit quod capacitas animae est finita substantia sed est capacitas infinita et sic negaret antecedens rationis nota rationes quasdam idem solvit
        \pend
     
        \pstart
        \ledsidenote{\textbf{42}}
        Ad aliud responsio quod deus posset creare unum obiectum ita delectabile quod si deus relinqueret animam naturae suae tantum delectaretur in eo sicut in deo et tamen non sequitur quod talis anima satiaretur vel beatificaretur in illo obiecto creato quia licet tantum delectaretur cum hoc tamen staret quod potius vellet illam delectationem habere de deo quam quam de obiecto creato
        \pend
     
        \pstart
        \ledsidenote{\textbf{43}}
        sed contra istam responsionem arguitur quia aut talis potest velle ita libenter habere delectionem de creatura sicut de creatore aut non si primum ergo satiatur in illo creato bono contra dicta si 2m ergo non aequaliter nec tantum delectatur in uno sic in alio cuius oppositum concedit casus
        \pend
     
        \pstart
        \ledsidenote{\textbf{44}}
        Item creatura magis inclinatur ad se quam ad deum ergo deus non est finis ultimus ad quem creatura movetur antecedens patet quia si creatura diligit deum hoc est ut bene sit sibi bene quidquid appetit est pro sui conservatione in esse et in bene esse
        \pend
     
        \pstart
        \ledsidenote{\textbf{45}}
        Item creatura rationalis est in potentia obiectiali ad quietandum in illo in quo deus vult eam quietari ergo finaliter potest quietari in bono creato tenet consequentia quia deus potest hoc velle et ordinate tum 2o quia deus potest suspendere motionem suam quam creatura movetur ad quietandum in deum
        \pend
     
        \pstart
        \ledsidenote{\textbf{46}}
        Concedo aliter dicitur quod nullum bonum citra deum animam potest satiare et beatificare eam quietare quia nullum bonum citra deum potest causare delectionem de illo genere quae foret quietativa ideo secundum hanc rationem quae est \name{hugolinus}\index[persons]{Hugolino of Orvieto} distinctione prima primi quaestione 3a circa finem posset negari consequentia
        \pend
     
        \pstart
        \ledsidenote{\textbf{47}}
        Item notandum quod licet creatura nequeat quietari finaliter in bono creato tamen potest credere quod sit quietata patet quia posset habere aliquod creatum diligibile sic quod credet esse ultimum appetible tamen decipitur propter malos actus
        \pend
     
        \pstart
        \ledsidenote{\textbf{48}}
        Item stat aliquem nescire deum esse et capere pro fine ultimo aliquod bonum temporale et \ledsidenote{L17v}tunc voluntas quietatur in illo et decipitur
        \pend
     
        \pstart
        \ledsidenote{\textbf{49}}
        ultimo posset dici quod deus potest creare aliquod bonum in quo anima potest tantum delectari et quietari prout li \enquote*{tantum} dicit gradum vel latitudinem individui in specie et non speciei in genere
        \pend
     
        \pstart
        \ledsidenote{\textbf{50}}
        Ad aliud contra 2am partem dicitur quod potest distingui quia vel talis diversitas rationis respicit motionem consurgentem ex subiecto et praedicamento et sic non est contra conclusionem vel respicit praecise obiectum et sic negatur antecedens et ad probationem quia obiectum etc conceditur et cum additur sed illa differunt ratione conceditur in creaturis ymmoimmo etiam differunt re sed non in deo ideo non habetur contra conclusionem quia in deo sunt omnino idem
        \pend
     
        \pstart
        \ledsidenote{\textbf{51}}
        Ad aliam probationem negatur consequentia licet forte concludat quod non est eadem ratio volendi et intelligendi sed non arguit distinctionem inter intelligibile et volibile
        \pend
     
        \pstart
        \ledsidenote{\textbf{52}}
        Ad aliam probationem primo posset negari consequentia et antecedens concedi  2o posset concedi consequens quia solum differentiam in creaturis arguit
        \pend
     
        \pstart
        \ledsidenote{\textbf{53}}
        Ad aliam probationem potest dici quod sicut malum culpae ut est difformitasdeformitas non habet rationem intelligibilitatis sicut nec volibilitatis et sic privatio non cognoscitur proprie nisi per habitum sic nec peccatum nisi ut est quemdam boni privatio nota \name{fastinum}\index[persons]{Fascinus of Asti} in 2o ubi de hoc ponit exemplum dicens quod homo scit se privari floreno non nisi quia in mente sua habet speciem floris super quem reflectitur sua ymaginarioimaginario et sic mente specie tali causatur iudicium de carentia floreni sic dicit ultra quod deus cognoscit peccatum non esse in \name{sorte}\index[persons]{Sortes} nec fore quia sic quod gratiam sine discontinuatione habebit
        \pend
     
        \pstart
        \ledsidenote{\textbf{54}}
        Ad aliud primo posset concedi consequens extendendo elicere ad causaliter efficere quia tunc fruitio includit actum intellectus et sic fruitio non est tantum actus voluntatis sed etiam intellectus  alio modo potest negari consequentia sumendo fruitionem stricte prout praecise dicit actum voluntatis tenendo rationem ex parte obiecti moventis
        \pend
     
        \pstart
        \ledsidenote{\textbf{55}}
        Ad ultimam rationem in oppositum primo dicerent \name{brauardinus}\index[persons]{Thomas Bradwardine} et \name{holkoth}\index[persons]{Robert Holcot} quod talis frueretur creatura sed excusaretur propter invincibilem errorem et ignorantiam absolutam  2o diceret \name{magister hugolinus}\index[persons]{Hugolino of Orvieto} quod talis non diligeret creaturam sed deum et ideo excusaretur potissime si in talem errorem non incidisset ex sua culpa
        \pend
     
        \pstart
        \ledsidenote{\textbf{56}}
        3o diceret \name{magister iohannes clenkot}\index[persons]{} quod talis excusaretur a tanto non a toto quia vel talis est in peccato originali vel non si primum tunc error ille est peccati pena nec excusatur si non sed est in veniali adhuc in penam peccati venialis deus iuste potest aliquem permittere in mortali cadere et sic non iterum excusatur quia non est verisimile quod aliquis sine quacumque culpa per mittatur a deo sic errare
        \pend
     
        \pstart
        \ledsidenote{\textbf{57}}
        veteris ac novae legis etc haec prima distinctio continet decem conclusiones prima quod tractatus sacrae paginae sive continentia veteris ac novae legis versatur praecise circa res et signa et vocat haec res quae non exhibentur ad significandum aliquid  signum vero dicitur quod ad aliquid significandi exhibetur signorum etiam quaedam sunt quibus non utimur nisi gratia significandi aliquid ut sunt sacramenta legalia quaedam quae non solum significant sed etiam ferunt quia intellectus adiuvant sicut sacramenta evangelica seu novae legis
        \pend
     
        \pstart
        \ledsidenote{\textbf{58}}
        2a conclusio quod in hoc libro \worktitle{sententiarum}\index[works]{} primo de rebus postea de signis \name{magister}\index[persons]{Peter Lombard} determinabit
        \pend
     
        \pstart
        \ledsidenote{\textbf{59}}
        3a conclusio quod secundum \name{augustinum}\index[persons]{Augustinus} triplex est genus rerum nam quaedam sunt quibus fruendum est ut pater \ledsidenote{L18r}filius et spiritus sanctus trinitas increata  quaedam quibus est utendum scilicet mundus et quae sunt in eo quaedam quae fruuntur et utuntur sicut nos et angeli sancti
        \pend
     
        \pstart
        \ledsidenote{\textbf{60}}
        4a conclusio quod frui est amore inhaerere alicui rei propter se ipsam uti vero est assumere aliquid in facultatem voluntatis  vel uti est illud quod venit in usum referre ad optinendi illud quo fruendum est
        \pend
     
        \pstart
        \ledsidenote{\textbf{61}}
        5a conclusio quod aliquo frui dicitur uti cum gaudio non adhuc speciei sed iam rei et sic omnis qui fruitur utitur quia assumit aliquid in facultatem voluntatis sed non econverso
        \pend
     
        \pstart
        \ledsidenote{\textbf{62}}
        6a conclusio quod frui perfecte plene et proprie est ubi videbimus quo fruemur scilicet in patria in hac autem vita fruimur sed imperfecte
        \pend
     
        \pstart
        \ledsidenote{\textbf{63}}
        7ma conclusio quod nullus se ipso debet frui quia nullus debet se diligere propter se sed propter illud quo fruendum est unde \name{paulus}\index[persons]{Paul, the Apostle} fruens \name{philomene}\index[persons]{} in domino potius deo fruebatur quam homine
        \pend
     
        \pstart
        \ledsidenote{\textbf{64}}
        8a conclusio quod deus diligit nos utendo nobis non fruendo ille autem usus conclusio deus nobis utitur non ad suam sed ad nostram refert utilitatem et ad cuius voluntatem tantum
        \pend
     
        \pstart
        \ledsidenote{\textbf{65}}
        9a conclusio quod virtutibus est bene utendum non fruendum ipse autem virtutes et ceterae potentiae animae sunt per quas fruimur et utimur
        \pend
     
        \pstart
        \ledsidenote{\textbf{66}}
        10 conclusio quod prius est tractandum de rebus quibus est fruendum id est de sancta ac individua trinitate et est sententia huius distinctionis
        \pend
     
        \endnumbering
        
     
        \end{document}
    
